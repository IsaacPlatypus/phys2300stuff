%% LyX 2.1.4 created this file.  For more info, see http://www.lyx.org/.
%% Do not edit unless you really know what you are doing.
\documentclass[12pt,english]{article}
\usepackage{charter}
\renewcommand{\sfdefault}{cmbr}
\usepackage[T1]{fontenc}
\usepackage[latin9]{inputenc}
\usepackage{setspace}
\usepackage[authoryear]{natbib}
\doublespacing

\makeatletter

%%%%%%%%%%%%%%%%%%%%%%%%%%%%%% LyX specific LaTeX commands.
%% Because html converters don't know tabularnewline
\providecommand{\tabularnewline}{\\}

%%%%%%%%%%%%%%%%%%%%%%%%%%%%%% User specified LaTeX commands.
\usepackage[hmargin=2.5cm,vmargin=2.5cm]{geometry}
\usepackage{layout}
\usepackage{indentfirst}

\makeatother

\usepackage{babel}
\begin{document}

\title{The Number of Nestle Smarties in a Ping-Pong-Ball-Contaminated Mason
Jar Encountered in the PHYS 2300 Lab}


\author{Anneya Golob}


\date{September, 2016}

\maketitle

\part*{Abstract \label{part:Abstract}}

The number of Nestle Smarties contained in a standard 500 mL Mason
jar which also held 3 ping pong balls suitable for international competition
was sought in the PHYS 2300 laboratory. Since it is against laboratory
rules to bring food or drink into the laboratory at any time, this
quantity was found by {[}\textbf{CONCISE EXPLANATION OF PROCEDURE}{]}.
The number of Nestle Smarties contained in the jar was found to be
$n_{\mathrm{Smarties}}\pm\sigma_{n_{\mathrm{Smarties}}}$ {[}\textbf{IN
AGREEMENT OR NOT?}{]} with the true value of 193 Smarties.


\section{Theory/Introduction\label{sec:Theory/Introduction}}

Nestle Smarties are a sugar-coated chocolate candy available in Canada
and numerous other countries. They are generally oblate spheroids
with a minor axis of approximately 5 mm and a major axis of approximately
12 mm, although individual candies with a wide range of defects have
been observed \cite{golob_2016}. 

During the first session of the PHYS 2300 lab, we competed in a contest
to accurately determine the number of Nestle Smarties, $n_{\mathrm{Smarties}}$,
in a Mason jar which also contained 3 ping pong balls. To aid in the
determination of the unknown quantity, a Fun Size box of Nestle Smarties
was provided (although - since food and drink are prohibited in the
lab - it proved to be of little use) in addition to an empty Mason
jar and tools to measure masses and distances...

The number of Nestle Smarties in a ping-pong-ball-contaminated Mason
jar can be expressed as 

\begin{equation}
n_{\mathrm{Smarties}}=\frac{m_{system}-m_{Mason\,jar}-3m_{ping\,pong\,Ball}}{m_{Smartie}}\label{eq:nSmarties}
\end{equation}


where $m_{Mason\,jar}$ is the mass of an empty Mason jar, $m_{ping\,pong\,ball}$
is the mass of a standard ping pong ball, $m_{Smartie}$ is the mean
mass of an individual Smartie candy, and $m_{system}$ is the mass
of the filled jar presented in the lab.


\section{Apparatus \& Procedure\label{sec:Apparatus-=000026-Procedure}}

To determine the number of Nestle Smarties in the Mason jar, all variables
in Equation \ref{eq:nSmarties} needed to be measured and recorded
with reasonable uncertainties. The mass of the filled jar, $m_{system}$,
and the mass of an empty jar, $m_{Mason\:jar}$, were determined using
a mechanical mass balance. The mass of a competition grade ping pong
ball and the mass of an individual Nestle Smartie were determined
by ... {[}WHAT DID YOU DO?{]}


\section{Data \& Analysis}

The results of the measurements and queries{[}?{]} described in Section
\ref{sec:Apparatus-=000026-Procedure} are described in Table \ref{tab:measurements}.
Uncertainties in the measured quantities were estimated by ... {[}INCLUDE
ANY REASONING THAT WENT INTO YOUR CHOICE OF UNCERTAINTIES{]}. The
values in Table \ref{tab:measurements} were used in Equation \ref{eq:nSmarties}
to determine the unknown number of Nestle Smarties in the Mason jar.

\begin{table}
\centering{}%
\begin{tabular}{|c|c|c|}
\hline 
Quantity & Value & Uncertainty \tabularnewline
\hline 
\hline 
$m_{system}$ & \# g & $\sigma_{m_{system}}$ g\tabularnewline
\hline 
$m_{ping\,pong\,ball}$ &  & \tabularnewline
\hline 
$m_{Mason\,jar}$ &  & \tabularnewline
\hline 
$m_{Smartie}$ &  & \tabularnewline
\hline 
\end{tabular}\caption{\label{tab:measurements}Measured values of quantities required to
determine the number of Nestle Smarties in a ping-pong-ball-contaminated
Mason jar.}
\end{table}



\section{Results \& Discussion}

The number of Nestle Smarties in the Mason jar was found to be $X\pm Y$
Smarties as demonstrated in Appendix \ref{sec:Sample-Calculations}.
The uncertainty in this measurement was dominated by the uncertainty
in the mass of {[}WHAT?{]} because {[}WHY?{]}. The true value of $n_{Smarties}$
was found to be 193, in {[}AGREEMENT/DISAGREEMENT?{]} with the estimated
value. The analysis performed in this lab was time-sensitive and limited
by our inability to legally isolate Nestle Smarties candies for individual
inspection because of laboratory regulations as outlined in the course
syllabus. Ideally, it would have been possible to {[}STUFF, THINGS{]}.


\section{Conclusions}

The number of Nestle Smarties contained in a ping-pong-ball-contaminated
Mason jar was determined during a 1 hour period in the PHYS 2300 lab,
where food and drink are prohibited. In light of the limitations of
the scenario, the number was determined by {[}VERY CONCISE EXPLANATION{]}
and found to be $\#\pm\#$, in {[}AGREEMENT/DISAGREEMENT{]} with the
true value of 193 Smarties. The accuracy and precision of the measurement
were hampered by {[}CIRCUMSTANCES{]}. To improve on the estimate of
$n_{Smarties}$, ... {[}WHAT YOU WOULD HAVE DONE TO DETERMINE THE
NUMBER OF SMARTIES IN THE JAR GIVEN UNLIMITED TIME AND MONEY{]}.

\begin{singlespace}
\bibliographystyle{aa}
\addcontentsline{toc}{section}{\refname}\bibliography{smarties}

\end{singlespace}

\appendix

\section*{Appendices}


\section{Derivation of Uncertainty in $n_{Smarties}$\label{sec:uncertainty}}

Assuming Gaussian uncertainties, the uncertainty in Equation \ref{eq:nSmarties}
can be derived as follows:

$\sigma_{n_{Smarties}}=\sqrt{\left(\frac{\partial n_{Smarties}}{\partial m_{system}}\right)^{2}\sigma_{m_{system}^{2}}+\left(\frac{\partial n_{Smarties}}{\partial m_{Mason\,jar}}\right)^{2}\sigma_{m_{Mason\,jar}}^{2}+\left(\frac{\partial n_{Smarties}}{\partial m_{ping\,pong\,ball}}\right)^{2}\sigma_{ping\,pong\,ball}^{2}+\left(\frac{\partial n_{Smarties}}{\partial m_{Smartie}}\right)^{2}\sigma_{Smartie}^{2}}$

Partial derivatives with respect to each measured can be expressed
as

$\frac{\partial n_{Smarties}}{\partial m_{system}}=$ ... ,

$\frac{\partial n_{Smarties}}{\partial Mason\,jar}=$ ...,

$\frac{\partial n_{Smarties}}{\partial{}_{ping\,pong\,ball}}=$...,
and

$\frac{\partial n_{Smarties}}{\partial Smartie_{system}}=$... .

Substituting the partial derivatives into the uncertainty equation
gives

\begin{equation}
\sigma_{n_{Smarties}}=\sqrt{\left(...\right)^{2}\sigma_{m_{system}^{2}}+\left(...\right)^{2}\sigma_{m_{Mason\,jar}}^{2}+\left(...\right)^{2}\sigma_{ping\,pong\,ball}^{2}+\left(...\right)^{2}\sigma_{Smartie}^{2}}.\label{eq:uncertaintyInNSmarties}
\end{equation}



\section{Sample Calculations\label{sec:Sample-Calculations}}

The number of Nestle Smarties in the Mason jar was calculated by substituting
the values in Table \ref{tab:measurements} into Equation \ref{eq:nSmarties}
as follows:

$n_{\mathrm{Smarties}}=\frac{X-Y-Z}{W}$

$n_{\mathrm{Smarties}}=\#$.

The uncertainty associated with the number of Nestle Smarties in the
Mason jar was calculated by substituting the values in Table \ref{tab:measurements}
into Equation \ref{eq:uncertaintyInNSmarties} as follows:

$\sigma_{n_{Smarties}}=\sqrt{\left(...\right)^{2}\sigma_{m_{system}^{2}}+\left(...\right)^{2}\sigma_{m_{Mason\,jar}}^{2}+\left(...\right)^{2}\sigma_{ping\,pong\,ball}^{2}+\left(...\right)^{2}\sigma_{Smartie}^{2}}$

$\sigma_{n_{Smarties}}=\#.$
\end{document}
